\documentclass[a4paper,10pt]{article}
\usepackage[utf8]{inputenc}
\usepackage[T1]{fontenc}
\usepackage[ngerman]{babel}
\usepackage{datatool}
\usepackage[freepin,crossmark]{ticket}
% To put a box around the nametags use:
%\usepackage[freepin,crossmark,boxed]{ticket}
\usepackage{fontspec}
\usepackage{xparse}
\setlength{\unitlength}{1mm}

% 3 rows and 3 columns on each page
\ticketNumbers{3}{3}
% size of each tag
\ticketSize{56}{85}
% size between the tags
\ticketDistance{4}{4}

\renewcommand*{\ticketdefault}{
}

\newcommand*{\front}[3]{%
  \ticket
  {
  	\put(12,50){\includegraphics[width=30mm]{logo.png}}
    \put(3,25){\resizebox{50mm}{!}{\bfseries #1}}
    \put(28,15){\makebox[0mm]{\bfseries\Large #2}}
    \put(28,5){\makebox[0mm]{\scshape\Large #3}}
  }
}

\NewDocumentCommand\back{>{\SplitList{;}}m}{%
	\ticket
	{
		\put(12,60){
			\parbox[b][4em][t]{0.33\textwidth}{
				\ProcessList{#1}{ \insertitem}
			}

    	}
    }
}

\newcommand\insertitem[1]{#1 \\}

\begin{document}
\sffamily

% Read data for frontside
\DTLloaddb{list}{example.csv}
\DTLforeach{list}{\first=first,\last=last,\info=info}{\front{\first}{\last}{\info}}

% Read data for backside
\backside%
\oddsidemargin=30mm
\DTLloaddb{groups}{example.csv}
% Loop through the lines of the csv file and extract the numbers of the groups.
\DTLforeach{groups}{
	\two=2x2x2,
	\rubiks=3x3x3,
	\blind=3x3 Blind,
	\oh=3x3x3 OH,
	\four=4x4x4,
	\five=5x5x5,
	\six=6x6x6,
	\seven=7x7x7,
	\pyra=Pyraminx,
	\mega=Megaminx,
	\clock=Clock,
	\sq=Square-1,
	\skewb=Skewb,
	\feet=Feet}
	{\back{
		2x2x2: \two;
		3x3x3: \rubiks;
		4x4x4: \four;
		5x5x5: \five;
		6x6x6: \six;
		7x7x7: \seven;
		One-Handed: \oh;
    	3x3x3 blindfolded:\blind;
		Pyraminx: \pyra;
		Skewb: \skewb;
		Rubik's Clock: \clock;
		Square-1: \sq;
		Megaminx: \mega;
		Feet: \feet;
		}
	}


\end{document}
